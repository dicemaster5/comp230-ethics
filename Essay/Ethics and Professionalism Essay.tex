% Please do not change the document class
\documentclass{scrartcl}

% Please do not change these packages
\usepackage[hidelinks]{hyperref}
\usepackage[none]{hyphenat}
\usepackage{setspace}
\usepackage{graphicx}
\usepackage{subcaption}
\doublespace

% You may add additional packages here
\usepackage{amsmath}

% Please include a clear, concise, and descriptive title
\title{What are the ethical implications for video game developers to make addictive games and how can developers help stop video game addiction?}

% Please do not change the subtitle
\subtitle{COMP230 - Ethics and Professionalism}

% Please put your student number in the author field
\author{1703086}

\begin{document}

\maketitle

\abstract{This essay is about video game addiction and how it is unethical for developers to be purposely designing games to be addictive. The essay proposed Humane Design as a better alternative to creating fun games and enjoyable games, Humane Design is a much more ethical way to design games as it's focus is to be good for the player.}

\section{Introduction}
%introduce the question and why it's being asked 
%talk about the issue
Video game addiction (also known as gaming disorder) can be a serious problem amongst certain gamers, there are many stories that you can find online on message boards about gamers who have gotten addicted to games and have suffered from it, some even claiming that the addiction has ruined their lives.
\\
\\
This essay will be looking at the ethical implications of developing addictive games, and looking into what makes a addictive game and who is to blame when a player becomes addicted to a game.  

\section{Video game addiction, what is it?}
%how do scientists and doctors define it, what do people do when they are addicted to video games?
The World Health Organization's definition for gaming disorder is "Gaming disorder is defined as a pattern of gaming behaviour characterized by impaired control over gaming, increasing priority given to gaming over other activities to the extent that gaming takes precedence over other interests and daily activities, and continuation or escalation of gaming despite the occurrence of negative consequences."
\\
In short video game addiction is a form of substance abuse, in which the player will keep playing a game or multiple games to the point that it starts negatively effecting their life and they will carry on playing despite being aware of the problems.\cite{Addiction1, Addiction2}


\section{What are the negative effects?}
%negative effects players get from falling into addictive behaviours when playing games: physical and mental strain, losing sleep, bad nutrition, failing daily life commitments
People who are heavily addicted to video games often endure many negative effects and consequences from playing games for long durations.\cite{Addiction3}
\\
Some of these effects include:
\begin{itemize}
  \item physical and mental strain
  \item loss of sleep
  \item failing daily life commitments
  \item Poor personal Hygiene
  \item Isolation from friends and family
  \item Feelings of dependency towards games
  \item obsessive-compulsive disorder
  \item depression
  \item anxiety
\end{itemize}
And in some serious cases it could even lead to death \cite{parkin2015death}.
\\
\\
Video game addiction can be a very serious matter just like any other kind of addiction and substance abuse.\cite{Addiction1, Addiction2, Addiction3}
\\
It would be highly unethical for a developer to be creating a product that can have these effects on a player, even if it's on a small number of players.
\\
\\
Developers should avoid creating games that could potentially impact players in these ways.\cite{Ethics1} Individuals who are addicted, need to find ways out, and learn to better understand how they might of gotten addicted to avoid getting addicted again.

\section{ways in which games are designed to be addictive}
%different ways in which games are designed to be addictive and keep player retention
In most cases when a developer creates a games they design it to be as fun and engaging as possible, in order to create a good experience for the player.
\\
\\
In other cases developers create games, designed in specific ways to hook the player with addictive game design methods, with the hope to retain as much of the player's attention in order to try and profit from the player through different means.\cite{AddictiveDesign1, AddictiveDesign2}
\\
Addictive design shouldn't be considered good design, as it's more like psychological traps meant to take advantage of human nature.


\subsection{Skinner Box Method}
%talk about the skinner box method and how it is applied in video games, give good examples.
%progression systems, exp, player stats and abbilties, RNG rewards, play cosmetics...etc
Skinner box methods in games are ways in which you could hook a player into wanting to play more using operant conditioning. \cite{SkinnerBox}
\\
\\
Things like: loot boxes, unlock-able rewards, collectibles, progression systems...etc play into being skinner box methods as they condition the player to want to play for longer in order to try and get these rewards.
\cite{AddictiveDesign4}

\subsection{Free to play and "casual" games}
Free to play games (F2P) are heavily designed in ways to keep the player hooked to the game often using Skinner box methods, in order to get the player seeing advertisements or buying into in-game items and currencies that cost real money.\cite{FreeToPlay} \cite{Ethics2}

\section{who's responsibility is it?}
%who's fault is it? player or developer, look at rat park
The responsibility for avoiding addiction should primarily fall onto the user as they are in charge of handling the use and time management of the product.
\\
\\
But the responsibility should also be given to the developers, as they decide on how the game might be designed. In the case of addictive games (games that are purposely designed with addictive hooks) developers should take responsibility for potentially harming individuals who develop addictions around the game that they created. These developers should be taking an ethical stand on how their game(s) might affect the player and create counter measures for players with addictive personalties to help them in not becoming addicted.
\\
\\
But there are also cases where it might be completely out of the developers control, as a player might develop an addiction to a game because they are looking for an escape.
\\
In the 1970's Bruce K. Alexander (a Canadian psychologist) and his colleagues did a study on rats known as the Rat Park Experiments, in which they experimented with the consumption of morphine amongst rats to see if they would get addicted under different circumstances, they found that a rat in a cage by itself with the choice to drink either water or water laced with morphine, would choose to drink the laced water and overdose on the morphine as it had nothing else to do, but a rat that was in a nicer cage that had balls and toys to play with and other rats to interact would almost never used the laced water and would never overdose.\cite{RatPark1, RatPark2}
\\
The conclusion of the experiments was that the individual's environment is the cause for addiction.
\\
\\
In the case of video games, this theory of addiction makes sense, if a player played video games as a form of escapism to escape from their bad or inadequate life to a fantasy life with limitless new possibilities, then it's no surprise that they would overplay and abuse a game, and give less care to their real life, bringing in the negative effects.
\\
There isn't much a developer can do if a player is stuck in a bad environment, but they can still take it into consideration and try to design and develop exits for the player using Humane Design.\cite{ExitPoints}

\section{Humane Design}
Humane Design in video games is all about designing a game that is good to the player. The goal of Humane Design is not to capture as much of the player's time as possible (unlike addictive game design) but instead the goal is to create a worthwhile, fulfilling and engaging experience.\cite{Addiction4, HumaneDesign}
\\
\\
Developing games with Humane Design and being aware of how it might affect the player(s), should in theory create games that are less likely to be addictive with more opportunities for the player to leave the game and go back to their lives, hopefully feeling more satisfied after playing the game instead of feeling dependent and or addicted towards the game.


\section{Conclusion}
%present solutions for game developers to help users not get addicted to their games or just generally play a healthy amount of time of the games.
%Humane design is my ultimate conclusion (https://www.youtube.com/watch?v=GArkyxP8-n0) 
%No it is not ethical for game developers to crteate purposley addictive games, game developers shouldn't be aiming to keep player retention at all times, what game developers should be aiming for is creating a worthwhile experience for the player, something engaging and fulfilling.
The main ethical implications that video game developers might get from developing addictive games is highly addicted players who are negatively affected in different ways from playing games, with the chance of ruining real human lives.
\\
\\
To avoid these ethical implications, developers should strive to create games using Humane Design in order to create a good experience for a player instead of an addictive escape.
\\
A player should feel satisfied and happier after playing a game rather than feeling stuck with no way out.


\bibliographystyle{ieeetran}
\bibliography{references}

\end{document}
